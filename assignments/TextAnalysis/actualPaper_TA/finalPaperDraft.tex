% !TeX root = ./finalPaperDraft.tex
\documentclass[titlepage]{article}
\author{Michael Lundquist}
\title{Text Analysis}

%-----------begin-preamble---------------------------
\usepackage{a4wide}										%wide use of a4 paper
\usepackage{apacite}									%bibliography in apa-style
\usepackage[doublespacing]{setspace}

\begin{document}
\maketitle

%====================================================
%-----------------references-------------------------
%====================================================
\bibliographystyle{apacite} %this was the key (apacann vs apacite)

\setlength{\bibleftmargin}{.125in}
%\setlength{\bibindent}{-\bibleftmargin}
\doublespacing

In “Creating and Evolving Developer Documentation: Understanding the Decisions of Open Source Contributions” Barthélémy Dagenais and Martin P. Robillard discuss the aspects of effective documentation. Using historical analysis and interviews, the authors determined how documentation is written and best practices for writing documentation. The historical analysis was primarily used to determine how documentation is currently written. Interviews were primarily used to determine pros and cons of different types of documentation. Although documentation methods differ for different types of projects, all projects benefit from good documentation in similar ways. Documentation can be a huge competitive advantage for a project. According to one of the interviewed contributors, "even if his project launched a year after a competing project, the user base grew quickly because the competing project had no documentation." \cite[p.6]{Dagenais:2010:CED:1882291.1882312}

Although this study provided insights into effective documentation practices, the study's qualitative methods were insufficient to draw solid conclusions. In the study's historical analysis, the authors "manually inspected more than 1500 revisions of 19 documents selected from 10 open source projects."\cite[p.1]{Dagenais:2010:CED:1882291.1882312}. Nineteen documents is hardly enough to draw serious conclusions from but can suggest patterns in how documentation is made that warrant future research. The project's authors interviewed 12 contributors to and 10 users of open source projects. These interviews were about a half-hour long. Again, half-hour interviews don't prove anything, but they do give insights into some expert's experiences developing documentations.

\section{Types of Documentation}

The study analyzed the pros and cons of different types of documentation. 

The first type of documentation the authors analyzed are wiki pages. Wiki pages, like wikipedia, are web pages that anyone can edit. As you would expect, allowing anyone to edit a wiki page can be a bad idea. First, because anyone can edit the wiki, they "lack authoritativeness" \cite[p.5]{Dagenais:2010:CED:1882291.1882312}. Furthermore, because wiki pages become "less concise" \cite[p.5]{Dagenais:2010:CED:1882291.1882312} and even suffer from SPAM. According to the article, "24.1\% of the revision in Firefox"\cite[p.5]{Dagenais:2010:CED:1882291.1882312} were SPAM. Because of these problems, "all of the projects we surveyed that started on a wiki (4 out of 12) moved to an infrastructure where contributions to the documentation are controlled"\cite[p.5]{Dagenais:2010:CED:1882291.1882312}. Despite these problems, wiki pages are extremely effective at encouraging community evolvement, which is critical to open source projects. To prevent anonymous actors from SPAMing the documentation, but still allow community involvement, the authors suggested using a comment section and encouraging feedback through various channels.

The second type of documentation analyzed was getting started documentation. Getting started documentation is designed to get users to a base level of competency with a tool as quickly as possible. If getting started documentation is well written, it can also serve as marketing material. Unfortunately, "Finding a good example on which to base the getting started documentation, an example that is realistic but not too contrived, is difficult" \cite[p.6 Contributor 11]{Dagenais:2010:CED:1882291.1882312}. Good getting started documentation is useful when learning frameworks because it gives the user a simple example that's consistent the framework's principles but doesn't overload the user.

The final type of documentation is reference documentation. Reference documentation describes 

\section{Documentation Management}

tools
writing docs immediately
writing a book (bursts)
separate Docs team
responsiveness to comments and questions


\bibliography{finalPaperDraft}
%\bibindent
%\bibleftmargin
% \nocite{*}



\end{document}