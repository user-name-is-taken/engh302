\documentclass[titlepage]{article}
\author{Michael Lundquist}
\title{The Perfect Storm}

%******* bibliography
\usepackage[backend=biber,style=apa]{biblatex}
%\usepackage{url, apacite}
%for some reason, url doesn't work with biblatex
\addbibresource{finalPaperDraft.bib}

%******* math
%https://tex.stackexchange.com/questions/41035/what-is-causing-undefined-control-sequence
\usepackage{amsmath}

%****** for double spacing
\usepackage{setspace}
%\singlespacing
%\onehalfspacing
\doublespacing
%\setstretch{1.1}

%****** titles
\usepackage[explicit]{titlesec}
\usepackage{ulem}
\usepackage{lipsum}% just to generate text for the example

\setcounter{secnumdepth}{4}
\titleformat{\paragraph}[runin]
  {\normalfont\normalsize\bfseries}{}{15pt}{\uline{\theparagraph\hspace*{1em}#1.}}
\titleformat{name=\paragraph,numberless}[runin]
  {\normalfont\normalsize\bfseries}{}{15pt}{\uline{#1.}}


% ******** graphviz highlighting def ********
\usepackage[pdf]{graphviz}
% ******** end graphviz highlighting def ********


\begin{document}

\maketitle

%*****************TITLE PAGE IS MADE************

Every programmer is ordained into the church of code with the same simple program. All it does is display the words "hello world." In 2014, I took a Computer Science class, printed "hello world", and am now forever a member of the church of code. When I wrote my first "hello world" program in 2014, I was already 21, rather old for a programmer. Being nearly a man when I first learned to code has given me a unique perspective on what life is like with and without programming and how fun it truly is.

\section{Before}

Before I learned to code, I occasionally faced problems that either required automation or could be performed much faster with automation. These problems didn't occur every day, but the problems that I did face heavily influenced and motivated me to keep working hard while learning how to code.


The problem that influenced me the most was given to me by my God father in an internship I did at his financial advisory firm in the summer of 2012, right after high school. At the internship, I was hired to copy information from thousands of websites and paste the information into an excel spreadsheet. At the time, automating the process didn't occur to me, but the task stuck with me and when I learned how to code, automating repetitive tasks became a very important to me. The task took months of hard work but if I were to do it again today, I would write web-scraper to automate all the copying and pasting. I would be done with months' worth of work in less than 10 hours.

When I started college in 2012, I was majored in Business as is applied most directly to my internship experience. In the Accounting class I took in my first year as a Business major, I learned the benefits of compound interest and how important it is to invest your money so it can grow. So I took the small amount of money I saved working part time jobs in high school and started investing it. Unfortunately, managing this money became a bit of a headache because I was always worried if I was investing in the right company. Eventually I concluded that a computer could help me manage these funds. Years later, I wrote a web scraper to find companies that changed more than 10\% in a day. I then investigated these companies to see if I should invest in them.

In my second semester of college, I took micro-economics. One of the core principles of micro-economics is opportunity cost. Simply put, opportunity cost states that different jobs pay different amounts, so you should choose the job that pays the most. At the same time, I had a close friend, Dimitri, who was a Computer Science major who would talk about how lucrative a career in this field could be. It took me a few years to realize that ultimately my micro-economics class was ironically telling me that Dimitri was right and I shouldn't be majoring in Business.


\section{During}

In 2014, on a whim, I decided to take my first coding class when I was still a business major. It was completely impractical as it didn't count toward my degree in any way. Despite its impracticality, the class could help me solve the problems described above. In addition to solving those old problems, new opportunities constantly presented themselves as I was learning to code. It quickly became an obsession that's shaped my life since.

As I was learning to code, I found a ton of interesting videos and other resources related my new skills. That semester I stayed up long nights drinking beer and watching videos of Defcon talks alone in my room. Even though I struggled to understand many of the topics in the videos, I dreamed of one day understanding them and advancing the field.

One day a friend invited me to a party where I met my girlfriend. She was a political science major and took the same coding class on a whim like me. She said she was having trouble in the class and I promised I would help her with it even though I was struggling in the class myself. In a desperate effort to impress her (which I'm still always trying to do), I learned quickly, got caught up in the class and passed what I learned on to her. 

\section{After}

Learning how to code was life changing. After I first learned how to code I knew it was something I wanted to do for the rest of my life the only question was "How?". So I switched from being a junior in the College of Business at the University of Mary Washington to being a Freshman Information Technology major at Northern Virginia Community College. Graduating years after my friends is a small price to pay for a rewarding career, financial stability and a girlfriend I love. 


These days I'm moving from the academic world of George Mason University to the professional field at Leidos. This summer I'll be returning to Leidos for an internship. In my previous experience at Leidos I learned how to work with developers using certain softwares. When I returned to George Mason, I used this experience to contribute to the Student Run Computing and Technology (SRCT) club at George Mason. Over this summer I look forward to introducing my team at Leidos to some of the technologies that SRCT uses for collaboration.  I intend to use the research I do on these collaboration technologies for the research paper in this class. Ultimately, the research paper will explore how teams of software developers can collaborate more effectively.



 % changing font
 %https://texblog.org/2012/08/29/changing-the-font-size-in-latex/





%******************REFERENCES***************
\begin{singlespace}


%https://www.latex-tutorial.com/tutorials/bibtex/
%https://www.youtube.com/watch?v=KS9GvK7cvmo
%https://tex.stackexchange.com/questions/134180/how-do-i-add-citations-at-the-end-of-the-document-as-done-here
%https://tex.stackexchange.com/questions/305381/biblatex-empty-bibliography
%\newpage
\nocite{*}
%\printbibliography
\end{singlespace}

\end{document}