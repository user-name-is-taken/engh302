\documentclass[titlepage]{article}
\author{Michael Lundquist}
\title{Research Interest Narrative - the perfect storm}

%******* bibliography
\usepackage[backend=biber,style=apa]{biblatex}
%\usepackage{url, apacite}
%for some reason, url doesn't work with biblatex
\addbibresource{finalPaperDraft.bib}

%******* math
%https://tex.stackexchange.com/questions/41035/what-is-causing-undefined-control-sequence
\usepackage{amsmath}

%****** for double spacing
\usepackage{setspace}
%\singlespacing
%\onehalfspacing
\doublespacing
%\setstretch{1.1}

%****** titles
\usepackage[explicit]{titlesec}
\usepackage{ulem}
\usepackage{lipsum}% just to generate text for the example

\setcounter{secnumdepth}{4}
\titleformat{\paragraph}[runin]
  {\normalfont\normalsize\bfseries}{}{15pt}{\uline{\theparagraph\hspace*{1em}#1.}}
\titleformat{name=\paragraph,numberless}[runin]
  {\normalfont\normalsize\bfseries}{}{15pt}{\uline{#1.}}



% ******** graphviz highlighting def ********
\usepackage[pdf]{graphviz}
% ******** end graphviz highlighting def ********


\begin{document}

\maketitle

%*****************TITLE PAGE IS MADE************

\section{Introduction}

Every programmer is ordained into the church of code with the same simple program. All it does is display the words 'hello world.' In 2014, I took a computer science class, printed 'hello world', and am now forever a member of the church of code. When I wrote my first 'hello world' program in 2014, I was already 21. 21 is old for a programmer to write their first 'hello world', being nearly a man when I first learned to code has given me a unique perspective on what life is like with and without programming and how fun it truly is. In this essay I will explain: the problems I had before I knew how to program, how much fun I had learning how to solve those problems, and how I currently program with a team of other expert programmers. Throughout this class and in my internship at Leidos this summer, I intend to further research tools that enable team work between developers. I hope to have another 'hello world' moment with some of these tools, and perhaps introduce some of them to my team!

\section{before}

Before I learned to code I occasionally faced many problems that either required code or could be performed much faster with code. These problems didn't occur every day, but the problems that I did face heavily influenced and motivated me to keep working hard while learning how to code.

\subsection{Joe}

The problem that influenced me the most was given to me by my God father in an internship I did at his financial advice firm in the summer of 2013. At the internship, I was hired to copy information from a thousands of websites and paste the information into an excel spreadsheet. At the time, automating the process didn't occur to me, but the task stuck with me and when I learned how to code, automating repetitive tasks became a very important to me. The task took months of hard work but if I were to do it again today, I would write web-scraper to automate all the copying and pasting. I would be done with months worth of work in less than 10 hours.

\iffalse
\subsection{St. Luke's}

Most people in IT are into it from a young age, not me\dots
\fi

\subsection{Business}

When I started college in 2012, I was initially a business major. I worked hard as a business major but I ran into certain problems where programming was simply the right solution.

\subsubsection{oportunity cost}

In my second semester of college, I took micro-economics. One of the core principles of micro-economics is opportunity cost. Simply put, opportunity cost states that different jobs pay different amounts, so you should choose the job that pays the most. At the same time, I had a close friend, Dimitri, who was a Computer Science major who would talk about how much money he would make after college. It took me a few years to realize that ultimately my micro-econoimcs class was ironically telling me that Dimitri was right and I shouldn't be majoring in business.

\subsubsection{investing}

In highschool I worked a few part time jobs and saved up a small amount of money. In the Accounting class I took in my first year as a business major, learned the benefits of compound interest and how important it is to invest your money so it can grow. Unfortunately, managing this money became a bit of a headache because I was always worried if I was invested in the right company. Eventually I came to the conclusion that a computer could help me manage these funds. Years later, I wrote a web scraper to find companies that changed more than 10\% in a day which I would then invest decide if I wanted in.

\section{during}

In 2014, on a whim, I decided to take my first coding class when I was still a business major. It was completely impractical as it didn't count toward my degree in any way. Despite it's impracticality, the class could help me in the solve the problems described above. In addition to solving those old problems, new opportunities constantly presented themselves as I was learning to code. Coding quickly became an obsession that's shaped my life since.

\subsection{Online resources}

As I was learning to code I found a ton of cool videos and other resources I could use to apply my new skills to once honed. That semester I stayed up long nights drinking beer and watching videos of Defcon talks alone in my room. Even though I struggled to understand many of the topics in the videos, I dreamed of one day understanding them and advancing the field. I'm not quite there yet. 

\subsection{Love interest}

One day a friend invited me to a party where I met my girlfriend. She was a political science major and took the same coding class as me on a whim like me. She said she was having trouble in the class and I promised I would help her with it even though I was struggling in the class myself. In a desperate effort to impress her (which I'm still always trying to do), I learned quickly got caught up in the class and passed what I learned on to her. 

\section{after}

Learning coding was a wild and life changing experience. 

\subsection{mom}

my mom said I'm too dumb for CS

\subsection{conclusion - propose a topic}

topic = DevOps hopefully (figure this out on monday)



\iffalse
\begin{itemize}
    \item readers-writers problem (requires waiting)
    \item producer consumer (1--1) (interrupts) vs consumer consumer (M--M)(roles) (the philosophers are both)
    \item readers-writers problem (requires waiting)
\end{itemize}
\fi



 % changing font
 %https://texblog.org/2012/08/29/changing-the-font-size-in-latex/





%******************REFERENCES***************
\begin{singlespace}


%https://www.latex-tutorial.com/tutorials/bibtex/
%https://www.youtube.com/watch?v=KS9GvK7cvmo
%https://tex.stackexchange.com/questions/134180/how-do-i-add-citations-at-the-end-of-the-document-as-done-here
%https://tex.stackexchange.com/questions/305381/biblatex-empty-bibliography
\newpage
\nocite{*}
\printbibliography
\end{singlespace}

\end{document}